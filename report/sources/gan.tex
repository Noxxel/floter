\chapter{Image Generation with GANs}
    In this chapter we elaborate on the different methods we employed to produce images from the extracted features. An arbitrary dataset of images can be used here, we went with \citetitle{102flower}~\cite{102flower}, a dataset of 102 different types of flowers containing slightly over 8000 images with a resolution of at least 500x500.
    
    \section{DCGAN}
        initially we trained on random input vectors of dimensionality 16/128/32 and got good results during training, blahblah. But because reasons that was shit. training with actually possible input configurations allows for a more reasonable mapping of the feature space to output images. as an example, include an image of how 4 different input vectors result in almost the same output img when gan is trained on random input! more blah.
        \begin{itemize}
            \item image of architecture
            \item mention loss, optimizer, ..
        \end{itemize}

    \section{InfoGAN}
        Idea is to now use a conditional GAN to introduce some random noise into the image generation while ensuring the latent code has a high impact on the image generation. Using random input together with the latent code may cause abrupt changes between subsequent images.
        \begin{itemize}
            \item image of architecture
            \item mention loss, optimizer, ..
        \end{itemize}

    \TODO{no results here, results will be discussed in chapter~\ref{ch:results}}