\chapter{Feature Extraction}
    In this chapter we discuss the different approaches for extracting features from the music. The basis for all our methods are the mel spectrograms.

\section{Mel Spectrograms}
    In sound processing, it is difficult to work with the raw signal of a song.
    Usually the signal is sampled 22050 to 44100 times per second and therefore yields a lot of data points to be processed.
    A way to counter this problem is the short-time Fourier transform. 
    It divides the signal into windows of equal size with a given amount of overlap, and then computes the Fourier transform for each of the individual windows.
    This results in the power of the frequencies which compose the amplitude of the signal, as they vary over time.\\
    Mel spectrograms are an extension of this method, which map the resulting powers to a logarithmic scale to better approximate the human perception of sound.

\section{Autoencoder}

\section{Convolutional Recurrent Neural Network}