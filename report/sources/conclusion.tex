\chapter{Conclusion}
    We reached our base goal of generating visualizations which visibly correlate to the music when played alongside.
    We achieved our best results with the generator which was trained on the raw mel spectrogram vectors.
    \TODO{
        \begin{itemize}
            \item maybe mimic the latent vector distribution instead of using pretrained models during training of GANs
            \item maybe should mention proGANs? slowly increasing the resolution seems smart
            \item train models where the discriminator is more complex than generator
        \end{itemize}
    }

    \section{Visualizing your own music}
    If the reader of this report would like to generate a visualization of a music track of his likings, he can clone the project repository from our github page (\TODO{INSERT LINK}) and download one of our pretrained state dictionaries from \TODO{INSERT LINK}. 
    After inserting the state into the correct \TODO{folder of the results subdirectory}, one can place their desired mp3 file in the $song\_in$ folder and execute the \TODO{main bash script}. 
    We currently only support the video generation directly from the mel spectrograms due to the mixed results of the other approaches.
    The script will then automatically calculate the mel spectrogram of the given song, apply a Gaussian smoothing over the time steps, feed the vectors to the pretrained generator and afterwords compile the images to a single video with the song added as the audio track.
    The resulting video will appear in the $song\_out$ folder adjacent to the $song\_in$ folder after everything is completed.